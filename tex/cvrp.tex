

\section*{Introduction}

The aim of this assignment is to design, implement and report local
search heuristic algorithms for solving the Capacitated Vehicle Routing
Problem (CVRP).



\medskip
Make sure you have read the whole document before you start to work.


\section*{Heuristics for Capacitated Vehicle Routing}

In vehicle routing problems we are given a set of \emph{transportation
  requests} and a fleet of \emph{vehicles} and we seek to determine a
set of \emph{vehicle routes} to perform all (or some) transportation
requests with the given vehicle fleet at \emph{minimum cost}; in
particular, we decide which \emph{vehicle handles which requests in which
  sequence} so that all \emph{vehicle routes} can be \emph{feasibly}
executed.

The \emph{capacitated vehicle routing problem} (CVRP) is the most studied
version of vehicle routing problems.

In the CVRP, the transportation requests consist of the distribution of
goods from a single \emph{depot}, denoted as point $0$, to a given set
of $n$ other points, typically referred to as \emph{customers},
$N=\{1,2,\ldots,n\}$. The amount that has to be delivered to customer
$i\in N$ is the customer's \emph{demand}, which is given by a scalar
$q_i\geq 0 $, e.g., the weight of the goods to deliver. The \emph{fleet}
$K=\{1,2,\ldots,|K|\}$ is assumed to be \emph{homogeneous}, meaning that
$|K|$ vehicles are available at the deport, all have the same capacity
$Q>0$, and are operating at identical costs. A vehicle that services a
customer subset $S\subseteq N$ starts at the depot, moves once to each
of the customers in $S$, and finally returns to the depot. A vehicle
moving from $i$ to $j$ incurs the \emph{travel cost} $c_{ij}$.

The given information can be structured using an undirected graph. Let
$V=\{0\}\cup N=\{0,1,\ldots,n\}$ be the set of \emph{vertices} (or
nodes). It is convenient to define $q_0:=0$ for the depot. In the
symmetric case, when the cost
for moving between $i$ and $j$ does not depend on the direction, i.e.,
either from $i$ to $j$ or from $j$ to $i$, the underlying graph
$G=(V,E)$ is complete and undirected with edge set $E=\{e=(i,j)=(j,i) :
i,j\in V,i\neq j\}$ and edge costs $c_{ij}$ for $\{i,j\} \in E$.
Overall, a CVRP instance is uniquely defined by a complete weighted graph
$G=(V,E,c_{ij},q_i)$ together with the size $|K|$ of the of the vehicle
fleet $K$ and the vehicle capacity $Q$.

A \emph{route} (or \emph{tour}) is a sequence
$r=(i_0,i_1,i_2,\ldots,i_s,i_{s+1})$ with $i_0=i_{s+1}=0$, in which the
set $S=\{i_1,\ldots,i_s\}\subseteq N$ of customers is visited. The route
$r$ has cost $c(r)=\sum_{p=0}^sc_{i_p,i_{p+1}}$. It is \emph{feasible}
if the capacity constraint $q(S):=\sum_{i\in S}q_i\leq Q$ holds and no
customer is visited more than once, i.e., $i_j\neq i_k$ for all $1\leq
j\leq k\leq s$. In this case, one says that $S\subseteq N$ is a
\emph{feasible cluster}. A solution to a CVRP consists of $K$ feasible
routes, one for each vehicle $k\in K$. The routes
$r_1,r_2,\ldots,r_{|K|}$ and the corresponding clusters
$S_1,S_2,\ldots,S_{|K|}$ provide a \emph{feasible solution} to the CVRP
if all routes are feasible and the clusters form a partition of
$N$. Hence, the CVRP consists of two interdependent tasks:
\begin{enumerate}[(i)]
\item the partitioning of the customer set $N$ into feasible clusters
  $S_1,\ldots,S_{|K|}$;
\item the routing of each vehicle $k\in K$ through $\{0\}\cup S_k$.
\end{enumerate}



\section{Your Tasks}
\medskip

Using the test instances described below, you have to submit a report and a
Python program that address the following tasks:


\begin{itemize}

\item Determine an easy-to-calculate lower bound $K_{LB}$ to the number
  of vehicles needed to satisfy the demand of all customers.  Report in
  a table of your final text document like the Table~\ref{tabexample}
  the lower bounds thus found for each given instance. (A way to obtain
  a lower bound, which is a bit harder to calculate than the one asked
  here, is by solving a problem addressed in the lectures; you are
  welcome to report about this procedure as well although this is not
  the main task of the assignment.)



\item Design and implement one or more construction heuristics.

\item Design and implement one or more iterative improvement
  algorithms. They must terminate in a local optimum.

\item Undertake an experimental analysis to compare and
  configure the algorithms from the previous two points. 

\item Describe the work done in a report of at most 10 pages. The report
  must at least contain a description of the best algorithm designed and
  the experimental analysis conducted. The level of detail must be such
  that it makes it possible for the reader to reproduce your work.

%Remember to
%  give details on the computational cost of the algorithms you designed.


\item In an appendix of the report (that does not count in the 10 pages)
  report the results of the best algorithms on the test instances
  made available (see below) in a table like Table~\ref{tabexample}. You
  are welcome to report also graphical comparisons and assessment of the
  way your algorithms scale with respect to the size of the instance
  (this must be included in the 10 pages).  




\begin{table}[tb]
\begin{center}
\begin{tabular}{|l||r|c|c|c|c|}
  \hline
Instance&
%\cline{2-3}
&\multicolumn{2}{c|}{Construction Heuristic} & \multicolumn{2}{c|}{Local Search}\\
& $K_{LB}$&cost&time (sec)&cost&time (sec)\\
  \hline
CMT01 &  &&&& \\
CMT02 &  &&&& \\
CMT03 &  &&&& \\
CMT04 &  &&&& \\
CMT05 &  &&&& \\
CMT06 &  &&&& \\
...   &  &&&& \\
...   &  &&&& \\
   \hline
\end{tabular}
\end{center}
\caption{\label{tabexample} The table shows the median results from 5 runs
  per instance of the best heuristic designed. The time limit was set to
  60 seconds on a Intel(R) Core(TM) i7-2600 CPU @ 3.40GHz with 16 GB RAM
  running Ubuntu 16.04.} 
\end{table}


\item Submit your best algorithm in the upload page. The programs will
  be run on a 64-bit machine with Ubuntu Linux, equivalent to those in
  the terminal room. A time limit of \textbf{60 seconds} will be
  imposed. If your algorithms are faster you can consider using some
  basic metaheuristic like random restart or neighborhood change. No
  other metaheuristic is allowed in this assignment.

\end{itemize}






\section*{Practicalities}

Associated to this document there is a GIT repository at:

\begin{center}
\url{https://github.com/DM865/CVRP}
\end{center}

The repository is made of a directory \verb!data/! containing the
instances, a directory \verb!src/! containing some initial Python 3 code
to read the instances, output a solution and produce a graphical view of
solutions. The code provides also a framework within which to organize
your implementation. The directory \verb!tex! contains the sources of
this document and can be therefore ignored.

\paragraph{Instances}
In the directory \verb!data/! you find the instance
\lstinline{A-n32-k05.xml} that is a small toy instance with 32 nodes.
This instance and a heuristic solution is represented in
Figure~\ref{A-n32-k05}.  In the directory \verb!data/CMT! you find the
set CMT\footnote{Christofides, N., Mingozzi, A., Toth, P. The vehicle
  routing problem. 1979.  In Christofides, N., Mingozzi, A., Toth, P.,
  Sandi, C. (Eds.), Combinatorial Optimization. Wiley, Chichester,
  pp. 315– 338.} of \emph{middle size} instances with number of nodes
ranging between 51 and 200, and in the directory \verb!data/Golden! you
find the set Golden\footnote{Golden, B. L., Wasil, E.A., Kelly, J.P.,
  Chao, I.-M. Metaheuristics in Vehicle Routing. 1998. In T. G. Crainic
  and G. Laporte, eds, Fleet Management and Logistics. Boston: Kluwer,
  pp. 33-56.} of \emph{large size} instances with number of nodes
ranging between 241 and 484. The displacement of the nodes in these
instances is depicted in Figure~\ref{CMTplace} and~\ref{Goldenplace}.
The best known solutions for these instances are reported in
Table~\ref{bks}.

%
%%custmers \cite{Christofides1969}.
%All instances are in the same XML format. The format allows you to
%inspect the instance if you need it. The names of the instances use the
%following scheme: \texttt{P-n[NNN]-k[KK].xml}, where \texttt{NNN} is the
%number of costumers and \texttt{KK} is the number of vehicles to use.
%

\begin{figure}
  \centering
  \includegraphics[width=0.5\textwidth]{figs/A-n32-k05_sol.png}
\caption{\label{A-n32-k05} A solution for the \texttt{A-n32-k05} instance}
\end{figure}

\begin{figure}
  \centering
  \includegraphics[width=0.24\textwidth]{figs/CMT01.png}
  \includegraphics[width=0.24\textwidth]{figs/CMT02.png}
  \includegraphics[width=0.24\textwidth]{figs/CMT03.png}
  \includegraphics[width=0.24\textwidth]{figs/CMT04.png}
  \includegraphics[width=0.24\textwidth]{figs/CMT05.png}
  \includegraphics[width=0.24\textwidth]{figs/CMT06.png}
  \includegraphics[width=0.24\textwidth]{figs/CMT07.png}
  \includegraphics[width=0.24\textwidth]{figs/CMT08.png}
  \includegraphics[width=0.24\textwidth]{figs/CMT09.png}
  \includegraphics[width=0.24\textwidth]{figs/CMT10.png}
  \includegraphics[width=0.24\textwidth]{figs/CMT11.png}
  \includegraphics[width=0.24\textwidth]{figs/CMT12.png}
  \includegraphics[width=0.24\textwidth]{figs/CMT13.png}
  \includegraphics[width=0.24\textwidth]{figs/CMT14.png}
\caption{\label{CMTplace} The CMT instances}
\end{figure}



\begin{figure}
  \centering
  \includegraphics[width=0.24\textwidth]{figs/Golden_01.png}
  \includegraphics[width=0.24\textwidth]{figs/Golden_02.png}
  \includegraphics[width=0.24\textwidth]{figs/Golden_03.png}
  \includegraphics[width=0.24\textwidth]{figs/Golden_04.png}
  \includegraphics[width=0.24\textwidth]{figs/Golden_05.png}
  \includegraphics[width=0.24\textwidth]{figs/Golden_06.png}
  \includegraphics[width=0.24\textwidth]{figs/Golden_07.png}
  \includegraphics[width=0.24\textwidth]{figs/Golden_08.png}
  \includegraphics[width=0.24\textwidth]{figs/Golden_09.png}
  \includegraphics[width=0.24\textwidth]{figs/Golden_10.png}
  \includegraphics[width=0.24\textwidth]{figs/Golden_11.png}
  \includegraphics[width=0.24\textwidth]{figs/Golden_12.png}
  \includegraphics[width=0.24\textwidth]{figs/Golden_13.png}
  \includegraphics[width=0.24\textwidth]{figs/Golden_14.png}
  \includegraphics[width=0.24\textwidth]{figs/Golden_15.png}
  \includegraphics[width=0.24\textwidth]{figs/Golden_16.png}
  \includegraphics[width=0.24\textwidth]{figs/Golden_17.png}
  \includegraphics[width=0.24\textwidth]{figs/Golden_18.png}
  \includegraphics[width=0.24\textwidth]{figs/Golden_19.png}
  \includegraphics[width=0.24\textwidth]{figs/Golden_20.png}
\caption{\label{Goldenplace} The Golden instances}
\end{figure}




\begin{table}[t]
\centering 
\begin{minipage}{0.48\textwidth}
\begin{tabular}{lrrr}
% & \multicolumn{2}{c}{tot.~length} & \\
instance & nodes &best known \\
\toprule
CMT01	& 51 &524.61*  \\
CMT02	& 76 &835.26*  \\
CMT03	& 101&826.14*  \\
CMT04	& 151&1028.42* \\
CMT05	& 200&1291.29* \\
CMT06	& 51 &555.43	  \\
CMT07	& 76 &909.68	  \\
CMT08	& 101&865.94	  \\
CMT09	& 151&1162.55  \\      
CMT10	& 200&1395.85  \\      
CMT11	& 121&1042.11* \\
CMT12	& 101&819.56*  \\      
CMT13	& 121&1541.14  \\      
CMT14	& 101&866.37	  \\
\bottomrule
\end{tabular}
\end{minipage}
%
\begin{minipage}{0.48\textwidth}
\begin{tabular}{lrrr}
% & \multicolumn{2}{c}{tot.~length} & \\
instance & nodes &best known \\
\toprule
Golden\_01 & 241 &	5623.47	\\
Golden\_02 & 321 &	8404.61	\\
Golden\_03 & 401 &	11036.22\\
Golden\_04 & 481 &	13592.88\\
Golden\_05 & 201 &	6460.98	\\
Golden\_06 & 281 &	8404.26	\\
Golden\_07 & 361 &	10102.68\\
Golden\_08 & 441 &	11635.34\\
Golden\_09 & 256 &	579.71	\\
Golden\_10 & 324 &	736.26	\\
Golden\_11 & 400 &	912.84	\\
Golden\_12 & 484 &	1102.69	\\
Golden\_13 & 253 &	857.19	\\
Golden\_14 & 321 &	1080.55	\\
Golden\_15 & 397 &	1337.92	\\
Golden\_16 & 481 &	1612.50	\\
Golden\_17 & 241 &	707.76	\\
Golden\_18 & 301 &	995.13	\\
Golden\_19 & 361 &	1365.60	\\
Golden\_20 & 421 &	1818.32	\\
\bottomrule
\end{tabular}
\end{minipage}
\caption{\label{bks} The best known solutions on the set of instances
  CMT and Golden. A star indicates that the solution has been proven optimal.}
\end{table}



\paragraph{Source Code}
The Python code in the directory \verb!src/! contains the following files: 

\begin{itemize}
\item \lstinline{data.py}\ \  that implements the class \lstinline{Data} to
  maintain the data associated with the input instance. It contains an
  instance reader for the XML format. Objects of this class contain the
  following data that will be relevant to you:
  \lstinline{capacity} giving the capacity of the vehicle and \lstinline{nodes}
that is a tuple container of the nodes of the given network. Each
element from the tuple \lstinline{nodes} is a dictionary with the
following keys and values: \verb!id!, the original identifier of the
node from the input file, \verb!pt!, the coordinates of the node in
complex numbers notation as we saw for the TSP, \verb!tp!, the type of
customer: 0 if a depot and 1 if a customer, \verb!rq!, the quantity
demanded by the node (if it is a depot this value is 0).
Nodes in the tuple \lstinline{nodes} are organized in such a way that
the depot is the first element followed by all others. Each node can be
accessed in constant time through the index in the tuple. Hence,
internally the depot has always index zero.     

The class \lstinline{Data} contains also methods for printing the
instance, reporting statistics, calculating distances and plotting.

\item \lstinline{solution.py}\ \  that implements the class \lstinline{Solution}
  to store data relative to a candidate solution. For now it assumes to
  store the solution in a list of lists, called
  \lstinline{routes}. However, this is up to changes according to your
  needs. It then assumes that you finish implementing the methods
  \lstinline{valid_solution} and \lstinline{cost} that determine the
  feasibility and the quality of the candidate solution.

The class \lstinline{Solution} contains also methods for writing the
solution file and for plotting the solution. These methods assume that
solutions are represented as lists of lists, where every inner list
representing a route starts with the depot and ends with the depot.
Note that the solution writer outputs the original identifier of the
nodes and not the one used to represent them internally.

\item \lstinline{solverCH.py}\ \  that implements the class
  \lstinline{ConstructionHeuristics}. Currently only a canonical
  construction is implemented that routes costumers in the order they
  are stored. The implementation might be helpful to see how to use the
  data from an object of class \lstinline{Data}.  
\item \lstinline{solverLS.py} that implements the class
  \lstinline{LocalSearch}.
\item \lstinline{main.py}\ \  that implements the main program defining the
  objects and calling the methods defined in the other files. It
  provides a starting example that can be modified at your best
  convenience. It also defines the parameters to be specified when the
  program is run. 
\end{itemize}


The program is executed as specified below:

\begin{lstlisting}
$ python3 main.py -h
usage: main.py [-h] [-o OUTPUT_FILE] -t TIME_LIMIT instance_file

positional arguments:
  instance_file   The path to the file of the instance to solve

optional arguments:
  -h, --help      show this help message and exit
  -o OUTPUT_FILE  The file where to save the solution and, in case, plots
  -t TIME_LIMIT   The time limit
marco@nat-102098:~/IMADA/DM865/CVRP/src$ 
\end{lstlisting}

for example: 
\begin{lstlisting}
python3 main.py -t 30 -o A-n32-k05 ../data/A-n32-k05.xml
\end{lstlisting}

Included in the directory there is also a \verb!Makefile! that can be
used to automatize tasks. For example, the call above can be also
achieved with:

\begin{lstlisting}
make A-n32-k05
\end{lstlisting}

In addition, in the Makefile there is an example on how to use the code
profiler: \verb!cProfile!.

It is possible to modify all these files and to add new ones.


\section*{Submission Guidelines}

The submission is done from \url{http://valkyrien.imada.sdu.dk/DOApp/}

You have to submit a tar gzip file.
Your directory must be organized as follows:
\begin{verbatim}
ob1
  |-  doc
  |-  src
\end{verbatim}
In the directory \verb=doc= put the report with your full name and
username. Keep it shorter than 10 pages.

In \verb=src= put all the source code. 
%make sure that it is the exact copy of the one delivered.
%
%Make such that once compiled your programs can be run with the following
%parameters:
%
%\begin{verbatim}
%./problem1 -file-sol ./sol.txt ../data/inst14.txt
%\end{verbatim}
%

You can then create the tar gzip file from the directory \verb!ob1/! as
follows:

\begin{lstlisting}
tar czvf ob1.tgz doc src 
\end{lstlisting}


You can submit as many times as you wish, each new submission overwrites
the previous one.

%Your programs will be
%compiled and run and the correctness of results will be checked.

To be considered acceptable, your source code must satisfy the following
requirements:

\begin{enumerate}[i)]
  \item it must execute the heuristic that you chose as the best one
     when called as
    follows:
\begin{lstlisting}
python3 main.py -t 30 -o [an_instance] ../data/[an_instance].xml 
\end{lstlisting}

The program must solve the specified instance and halt before the
specified time limit.

\item At termination the program must write the solution in the format
  described below in a file whose name is the one given for the
  parameter \verb!-o! plus the extension \verb!.sol!. The starting code
  provided has a function \lstinline{write_to_file} to do this but
  probably you will need to modify it if you change the solution
  representation. The function is called from the main file.
  The solution written in the file must be valid, that is, feasible.
\end{enumerate}

  

Right after the submission the program will be tested and if it does not
satisfy the requiremets above you will receive an email and the submission
will be invalid. 

In addition, the submission system will execute your program and compare
it against your peers on a set of unspecified instances. Therefore, you should
submit your best algorithm early and eventually revise your submission.
Before submitting, test your implementation on the IMADA machines.
If you are using additional python modules not present in setting of the
Computer Lab machines, write to Marco.


\paragraph{Solution file} The solution file must list the routes one per
line. Each route is a comma separated list of nodes to be visited in the
given order. The node identifier must be the original one from the input
file.  Routes must start with the depot and finish with the depot.

The following listing provides an example of solution file for a valid
solution to the instance CMT02:

\begin{lstlisting}
76,1,2,3,4,5,6,7,76
76,8,9,10,11,12,13,76
76,14,15,16,17,18,19,20,76
76,21,22,23,24,25,26,27,76
76,28,29,30,31,32,76
76,33,34,35,36,37,38,39,76
76,40,41,42,43,44,45,76
76,46,47,48,49,50,51,52,76
76,53,54,55,56,57,58,59,76
76,60,61,62,63,64,65,66,76
76,67,68,69,70,71,72,73,74,75,76 
\end{lstlisting}

Solutions files have extension \verb!.sol!.



\section*{Remarks}

\begin{remarks} 


\item This is a list of factors that will be taken into account in the
evaluation:


\begin{itemize} 
\item quality of the final results;
\item level of detail of the study;
\item complexity and originality of the approaches chosen;

\item organization of experiments that guarantees reproducibility of
  conclusions;
\item clarity of the report;
\item presence of the analysis of the computational costs involved in
  the main operations of the local search.

\item effective use of graphics in the presentation of experimental results.
\end{itemize}


\item 
%In the project requirements 
%of Sec.~\ref{requirements} the words
%  ``one or more algorithms'' do NOT imply 
% ``the more algorithms, the better the final grade''. 
Note that a few, well thought algorithms are better than many naive ones!


\color{red}

\item If you search on Internet, the literature on heuristics for
  vehicle routing problems is vast but not every article is . The following are two relevant
  articles:


\begin{itemize}


\item Clarke, G., and J. W. Wright. "Scheduling of Vehicles from a
  Central Depot to a Number of Delivery Points." Operations Research 12,
  no. 4 (1964): 568-81. \url{http://www.jstor.org/stable/167703.}

\item Ann Melissa Campbell and Martin Savelsbergh, Efficient Insertion
  Heuristics for Vehicle Routing and Scheduling Problems, Transportation Science,
38(3), 369-378, 2004, \url{http://dx.doi.org/10.1287/trsc.1030.0046},

  
\item G. Kindevater and M. Savelsbergh, "Vehicle routing 2 - handling
  side constraints," tech. rep., School of Industrial and Systems Eng.,
  Georgia Institute of Tech., 1995.

  
\end{itemize}

\end{remarks}


\end{document}
