
\begin{itemize}

\item This is the \emph{first obligatory assignment} of Heuristics and
  Approximation Algorithms. It will be graded and it will contribute for
  50\% of the final grade.

\item  The assignment has to be carried out individually.

\item The submission is electronic via
  \ahref{\url{http://valkyrien.imada.sdu.dk/DOApp/}}{\url{http://valkyrien.imada.sdu.dk/DOApp/}}.

\item  You have to hand in:
  \begin{itemize}
  \item the source code of your implementation of a heuristic
    solver. Submit all your files in a \lstinline{.tgz} archive. Your
    code will be compiled and run, hence it must comply to the
    requirements listed in this document. 
    
  \item A report that describes the work you have done and presents the
    results obtained. The document should not exceed 10 pages and must
    be in PDF format. You cannot list source code. You can write in
    Danish or in English.
  \end{itemize} 

\item Make your submissions anonymous. The submission system will rename
  your files with the identifier that you will see on the page and your
  identity will be retrieved after the grading has been done.


\item Changes to this document after its first publication on December
  19 may occur. They will be emphasized in color and if they are major
  they will be announced via BlackBoard.

\item Read all this document before you start to work.

\item A starting package containing the instances and the code to read
  them is available at this link:
\begin{center}
  \ahref{\url{http://www.imada.sdu.dk/~marco/DM841/Files/BP.tgz}}{\url{http://www.imada.sdu.dk/~marco/DM841/Files/BP.tgz}}
\end{center}
\end{itemize}




\section*{Introduction}

The aim of this final assignment is to design, implement and report local
search heuristic algorithms for solving



\medskip
Make sure you have read the whole document before you start to work.


\section*{Heuristics for Capacitated Vehicle Routing}

In vehicle routing problems we are given a set of \emph{transportation
  requests} and a fleet of \emph{vehicles} are we seek to determine a
set of \emph{vehicle routes} to perform all (or some) transportation
requests with the given vehicle fleet at \emph{minimum cost}; in
particular, decide which \emph{vehicle handles which request in which
  sequence} so that all \emph{vehicle routes} can be \emph{feasibly}
executed.

The \emph{capacitated vehicle routing problem} (CVRP) is the most studied
version of vehicle routing problems.

In the CVRP, the transportation requests consist of the distribution of
goods from a single \emph{depot}, denoted as point $0$, to a given set
of $n$ other points, typically referred to as \emph{customers},
$N=\{1,2,\ldots,n\}$. The amount that has to be delivered to customer
$i\in N$ is the customer's \emph{demand}, which is given by a scalar
$q_i\geq 0 $, e.g., the weight of the goods to deliver. The \emph{fleet}
$K=\{1,2,\ldots,|K|\}$ is assumed to be \emph{homogeneous}, meaning that
$|K|$ vehicles are available at the deport, all have the same capacity
$Q>0$, and are operating at identical costs. A vehicle that services a
customer subset $S\subseteq N$ starts at the depot, moves once to each
of the customers in $S$, and finally returns to the depot. A vehicle
moving from $i$ to $j$ incurs the \emph{travel cost} $c_{ij}$.

The given information can be structured using an undirected graph. Let
$V=\{0\}\cup N=\{0,1,\ldots,n\}$ be the set of \emph{vertices} (or
nodes). It is convenient to define $q_0:=0$ for the depot. In the
symmetric case, when the cost
for moving between $i$ and $j$ does not depend on the direction, i.e.,
either from $i$ to $j$ or from $j$ to $i$, the underlying graph
$G=(V,E)$ is complete and undirected with edge set $E=\{e=(i,j)=(j,i) :
i,j\in V,i\neq j\}$ and edge costs $c_{ij}$ for $\{i,j\} \in E$.
Overall, a CVRP instance is uniquely defined by a complete weighted graph
$G=(V,E,c_{ij},q_i)$ together with the size $|K|$ of the of the vehicle
fleet $K$ and the vehicle capacity $Q$.

A \emph{route} (or \emph{tour}) is a sequence
$r=(i_0,i_1,i_2,\ldots,i_s,i_{s+1})$ with $i_0=i_{s+1}=0$, in which the
set $S=\{i_1,\ldots,i_s\}\subseteq N$ of customers is visited. The route
$r$ has cost $c(r)=\sum_{p=0}^sc_{i_p,i_{p+1}}$. It is \emph{feasible}
if the capacity constraint $q(S):=\sum_{i\in S}q_i\leq Q$ holds and no
customer is visited more than once, i.e., $i_j\neq i_k$ for all $1\leq
j\leq k\leq s$. In this case, one says that $S\subseteq N$ is a
\emph{feasible cluster}. A solution to a CVRP consists of $K$ feasible
routes, one for each vehicle $k\in K$. The routes
$r_1,r_2,\ldots,r_{|K|}$ and the corresponding clusters
$S_1,S_2,\ldots,S_{|K|}$ provide a \emph{feasible solution} to the CVRP
if all routes are feasible and the clusters form a partition of
$N$. Hence, the CVRP consists of two interdependent tasks:
\begin{enumerate}[(i)]
\item the partitioning of the customer set $N$ into feasible clusters
  $S_1,\ldots,S_{|K|}$;
\item the routing of each vehicle $k\in K$ through $\{0\}\cup S_k$.
\end{enumerate}



\section{Your Tasks}
\medskip

Using the test instances described below, you have submit a report and a
source code that address the following tasks:


\begin{itemize}

\item Determine an easy-to-calculate lower bound $K_{LB}$ to the number
  of vehicles needed to satisfy the demand of all customers. Report in a
  table of your final text document the lower bounds thus found for each
  given instance. (Another way to obtain a lower bound, which is a bit
  harder to calculate, is by solving a problem addressed in one of the
  examples of Gecode and described in the manual [MPG, part C] --
  \color{blue}you will need this to address point 3 below.\color{black})

\item Model the problem in constraint programming terms and describe the
  model in the report. \color{black} Ignore for the moment the objective
  of minimizing the total length of the routes and focus on feasibility
  only.\color{black}


\item Implement the model at point 2.~above in Gecode. Let $K^*$ be the
  minimum number of vehicles needed cover the demand of all customers of
  an instance. Determine this number for all instances given and compare
  it with $K_{LB}$. Report in a table the comparison with the lower
  bounds found in point 1.

\item Modify your model to minimize the total length of the routes while
  using the number of vehicles given in the name of the instance. Report
  the results in a table. Indicate whether the instance is solved to
  optimality or not and report meaningful measures to describe the
  effort needed to find the solutions. If the solution found is not
  optimal within a given time limit, report the best solution found so
  far.
  
\item Improve your model by fine tuning some parameters of your choice
  and present the analysis with numerical results and discussion. The
  parameters can be: branching rules, type of consistency, redundant
  constraints, alternative models, symmetry breaking.

\item Study the use of random restarts. Describe the restart strategy and
  the configuration you chose and find out whether random restarts
  improve or not the performance of the solver. See sec.~9.4 from [MPG]
  for instructions on how to do random restart. Start your
  implementation from the best model found so far, but make sure that a
  restart is worth by ensuring that at each restart a modified problem
  is solved. Consider using no-goods, shared heuristics or random
  choices or different heuristics at each restart.
  
\item Implement a large neighborhood search to improve the quality of
  the best solutions found in the largest instances. Describe your
  attempt in the report.
\end{itemize}


 
There is no predefined time limit. You can choose to run your
experiments with the time limit that best fit with your computation
resources (eg, time left from deadline, number of cores and machines,
etc).  As always the final results count. Performance can be assessed in
several ways (max size of the instances solved, computation time,
solution quality, etc.) It is up to you to identify meaningful measures
and comment on performance.





All instances are \emph{minimization} problems. The ones that will be
used to test your solvers are listed in Table~\ref{instances}.




\section*{Practicalities}

Associated to this document there is a starter package
\ahref{\url{http://www.imada.sdu.dk/~marco/DM841/Resources/CP3/vrp.tgz}}{\texttt{vrp.tgz}}
containing the test instances and some initial code to read the
instances, output a solution and produce a graphical view of the
instances and solutions in SVG format (see example in Figure~\ref{toy}).

\paragraph{Instances}
In the directory \texttt{data/augerat/P} there is one main set of
instances of varying size, from 16 to 101 customers. This is the set P
from \cite{Augerat1995}. In addition, you are given a small instance
with 10 costumers \texttt{toy.xml}. The toy instance and a solution for
it is visualized in Figure~\ref{toy}.
%custmers \cite{Christofides1969}.
All instances are in the same XML format. The format allows you to
inspect the instance if you need it. The names of the instances use the
following scheme: \texttt{P-n[NNN]-k[KK].xml}, where \texttt{NNN} is the
number of costumers and \texttt{KK} is the number of vehicles to use.

\paragraph{The program}

Enter in the directory \texttt{src/} and edit the makefile to make sure
that the path to the Gecode library is correct.  Compile the program via
\texttt{make} and execute
\begin{verbatim}
$ xyz ../data/toy.xml 
\end{verbatim}
This will read the instance and use a stored solution for it. It will
print in the standard output some relevant data of the instance and
produce a file \texttt{routes.svg} which can be used to visualize the
instance and the solution with any browser. For example in Linux:
\begin{verbatim}
$ firefox routes.svg
\end{verbatim}
In OSX you can use the command \texttt{open}.  

\color{blue}
If parameters have to be passed by command line, they must be set before
the name of the instance file. For example:
\begin{verbatim}
$ xyz -time 60000 ../data/toy.xml 
\end{verbatim}
\color{black}


\bigskip
The source files \texttt{tinyxml2} contain the library to read XML
files. You do not need to edit these files. The files \texttt{Data}
contain the class \texttt{Input} and \texttt{Output}.
The class  \texttt{Input} contains the relevant data of the instance, precisely:


\begin{itemize}
\item \verb=int decimals= this parameter defines how to approximate the
  calculation of the length of the arcs. You should not need it;
\item \verb=const char* name;= the name of the instance;
\item \verb=bool symmetric;= whether the arcs are symmetric on not. For
  the given instances the graphs are undirected and complete. In the
  general case the graph may also be directed and incomplete;
\item \verb=vector<_vehicle> vehicles;= The set of vehicles with a type,
  a departure node, an arrival node and the capacity. You will not need
  to use the type in this assignment;
\item \verb=vector<int> depots;= a list of nodes that represent
  depots. In the given instances there is only one depot. Note that this
  is not necessarily the node 0.
\item \verb=map<int, _node> nodes;= The set of nodes. The key of the map
  is the identifier of the node. The structure \verb=_node= contains the
  id, the type, the demand, and the coordinates cx and cy. The type
  defines whether the node is a depot ($0$) or not ($\neq 0$);
%If not it
%  determines the type of vehicle that can service the costumer. In the
 % instance given all vehicles can service all costumers;
\item \verb=unordered_map<arcKey, _arc, arcKeyHash, arcKeyEqual> arcs;=
  The set of arcs. The map has as key a pair of numbers (tail,head)
  representing the nodes of the arc. The structure \verb=_arc= defines
  the tail (starting node), the head (ending node) and the length;
\end{itemize}

The class \texttt{Output} contains \verb=map<int, list<int> > routes=,
that is, a set of lists of costumers representing the routes. Each of
these lists must start and end with the depot. The class also contains
the function to plot the instance and the overwriting of the output
operator to write the solution in the format required in the standard
output.

You will need to modify the function \color{blue}
\verb=setSolution(Output & o)= \color{black} to
process the solution found by your model and write it to
\verb=Otuput::routes=. The function is then called by in the
\verb=main()= function:

\begin{verbatim}
out.getSolution();
cout << out << endl;
out.draw();
\end{verbatim}

The second line above prints the solution in text form in the standard
output while the third row produces the plot. You can avoid using this
function if you encounter problems.

\medskip

You will have to modify the \texttt{main}.cpp file. The code provided
in \texttt{main.cpp} checks that a given time limit is not exceeded
during the search. This is achieved by means of a stop object. See
sec.~9.3.2 from [MPG].




In Table ~\ref{results} you find a list of benchmark results:
  namely, the optimal solutions and the results obtained by a CP model
  without too much fine tuning after 60 seconds.

\begin{table}[t]
\centering 
\color{blue}
\begin{tabular}{lrrr}
 & \multicolumn{2}{c}{tot.~length} & \\
instance & nodes & |K| &best known \\
\toprule
A-n32-k05 &\\
CMT01	&524.61*  \\
CMT02	&835.26*  \\
CMT03	&826.14*  \\
CMT04	&1028.42* \\
CMT05	&1291.29* \\
CMT06	&555.43	  \\
CMT07	&909.68	  \\
CMT08	&865.94	  \\
CMT09	&1162.55  \\      
CMT10	&1395.85  \\      
CMT11	&1042.11* \\
CMT12	&819.56*  \\      
CMT13	&1541.14  \\      
CMT14	&866.37	  \\
\hline
%Golden_01	5623.47	
%Golden_02	8404.61	
%Golden_03	11036.22
%Golden_04	13592.88
%Golden_05	6460.98	
%Golden_06	8404.26	
%Golden_07	10102.68
%Golden_08	11635.34
%Golden_09	579.71	
%Golden_10	736.26	
%Golden_11	912.84	
%Golden_12	1102.69	
%Golden_13	857.19	
%Golden_14	1080.55	
%Golden_15	1337.92	
%Golden_16	1612.50	
%Golden_17	707.76	
%Golden_18	995.13	
%Golden_19	1365.60	
%Golden_20	1818.32	
\bottomrule
\end{tabular}
\caption{\label{results} The time limit used was 60 seconds. A start
  indicates that the solution has been proved optimal.}
\end{table}
 

%\end{itemize}


\section*{Submission Details}

The submission is done from the BlackBoard system via SDU Assignment.
Your directory must be organized as follows:
\begin{verbatim}
ob1
  |-  doc
  |-  src
\end{verbatim}
In the directory \verb=doc= put the report with your full name and
username. Keep in shorter than 12 pages.

In \verb=src= put all the source code. Include a \verb=Makefile=.
%make sure that it is the exact copy of the one delivered.
%
%Make such that once compiled your programs can be run with the following
%parameters:
%
%\begin{verbatim}
%./problem1 -file-sol ./sol.txt ../data/inst14.txt
%\end{verbatim}
%


You can submit as many times as you wish. 

%Your programs will be
%compiled and run and the correctness of results will be checked.



\clearpage


The assignment has a
\textbf{starting package} that has already implemented a parser of the
instances. The organization is the same as the packages that you had to
submit in the previous assignment.

\begin{verbatim}
BP_ls/ 
|-- bin
|-- data
|-- Makefile
|-- make.local
|-- src
\end{verbatim}

In \verb=data/= you find a few instances for testing and training your
solver. There is also a very small instance \verb=toy.mps= for debugging
purposes. Finally, the directory \verb=src/= contains the source code
that you have to expand.





\section*{Requirements}
\label{requirements}


You are asked to carry out the following tasks: 

\begin{enumerate}

\item Design and implement stochastic local search or
  metaheuristic algorithms that comprise a construction heuristic and a
  local search.

\item Undertake an experimental analysis where you compare and
  configure the algorithms from the previous point. 

\item Describe the work done in a report of at most 10 pages. The report
  must at least contain a description of the best algorithm designed and
  the experimental analysis conducted. The level of detail must be such
  that it makes it possible for the reader to reproduce your work.

%Remember to
%  give details on the computational cost of the algorithms you designed.


\item Report the results of the best algorithms on the test instances
  made available in a table like Table~\ref{res}.




\begin{table}[tb]
\begin{center}
\begin{tabular}{|l||r|r|}
  \hline
Instance&
\multicolumn{2}{c|}{Your Heuristics}\\
\cline{2-3}
& Construction Heuristic&Metaheuristic\\
  \hline
acc-tight4&&\\
air04&&\\
cov1075&&\\
n3seq24&&\\
neos18&&\\
seymour&&\\
sp97ar&&\\
tanglegram1&&\\
   \hline
\end{tabular}
\end{center}
\caption{\label{res} The table shows the median results from 5 runs
  per instance of the best heuristic designed. The time limit was set to
  120 seconds on a Intel(R) Core(TM) i7-2600 CPU @ 3.40GHz with 16 GB RAM
  running Ubuntu 14.04.} 
\end{table}


\item Submit your best algorithm in the upload page. The programs will
  be run on a 64-bit machine with Ubuntu Linux, equivalent to those in
  the terminal room. A time limit of \textbf{120 seconds} will be
  imposed.


\end{enumerate}

All the points below must be addressed in any order:

\begin{enumerate}

\item Model the satisfiability problem in terms of local
  search 

\item Design a basic \emph{iterative improvement} algorithm with a
  natural termination in a local optimum. Call this algorithm: \texttt{sat\_ii}.

\item Design speedups for the execution of \texttt{sat\_ii} and call
  the new algorithm \texttt{sat\_ii\_+}.

\item Design stochastic local search algorithms that go beyond local
  optima and that do not have a natural termination. Examples to
  consider are: (probabilistic iterative improvement, min conflict
  heuristic, simulated annealing, tabu search, variable neighborhood
  search, etc.) Call these algorithms \texttt{sat\_sls\_*}, where the
  star indicates a suffix of your choice.


\item Implement \texttt{sat\_ii}, \texttt{sat\_ii\_+} and
  \texttt{sat\_sls\_*}. For this you can use the start package linked below
  based on EasyLocal. 

\item Execute the algorithm \texttt{sat\_ii} 30 times and plot the
  distribution of running times and of solution quality.

\item Show experimentally that \texttt{sat\_ii\_+} is indeed more
  efficient than \texttt{sat\_ii}.

\item Compare experimentally your algorithms from point 4.~on the given
  instances when run with a time limit of 20 seconds on a desktop
  computer equivalent to those in the IMADA terminal room. For each
  algorithm collect at least 10 runs per instance. Include in the
  comparison a baseline algorithm consisting in a random restart of your
  \texttt{sat\_ii\_+}.

\item Describe in a short report (max 5 pages):
  \begin{itemize}
  \item your model for your local search approach to SAT
  \item all your algorithms \texttt{sat\_ii}, \texttt{sat\_ii\_+} and
    \texttt{sat\_sls\_*} to a level of detail that allows replication
  \item the speedups introduced 
  \item the experimental analysis
  \end{itemize}
    Use proper mathematical formalism and computer science sketches to
    specify your algorithms. Use R for the analysis of results and the
    production of graphics and tables.


\end{enumerate}


In addition, the submission system will execute your program and compare
it against your peers on new unseen instances. Therefore you should
submit your best algorithm early and eventually revise your submission.
Before submitting, test your implementation on the IMADA machines.



\section*{Remarks}

\begin{remarks} 


%\item The assignment has to be carried out individually and it is not
%  allowed to collaborate.


\item The assignment is a revision and development of the previous
  preparatory assignment. Therefore, it contains modelling the problem
  in terms of local search, design of basic local search algorithms and
  metaheuristics, implementation, experimentation and written report.
  It is expected that the feedback provided in the previous assignment
  is used in this assignment.

%
  Differently from the previous two assignments on local search, the
  submission will be graded with the 7-step scale and external
  censorship.  The evaluation of the project is based on the report and
  the results that will appear on the web page.

\item You have to submit:

  \begin{itemize}
  \item Within the Deadline: the source code that compiles an executable
    program that implements the best algorithm developed and returns
    valid solutions within the time limit; See the Appendix \ref{a1} and
    \ref{a2} of this document for the specifications of this
    submission. The text of Appendix \ref{a1} is copied from the
    previous assignment. Note that you do not have to submit the
    directory \verb=\data=.


  \item Within the Deadline: a written report that may be written
    in Danish or in English. 
    %The report in PDF format has to be
    %submitted also via the BlackBoard system under SDU Assignments from the
    %left menu.
    \end{itemize} 

    Note that the deadline is strict. Codes and reports handed in after
    the deadline will generally not be accepted. System failures,
    illness, etc.\ will not automatically give extra time.
 
 
\item Corrections or updates to the project description, if any, will be
  published on the course web page and will be announced by email to the
  addresses available in the BlackBoard system. In any case, it remains
  students' responsibility to check for new announcements.


%\item Write your name and your user ID as it appeared in the page of
%  results in the front page of the report.
%
%\item The evaluation of the submissions will be communicated to the exam
%  office not later than three weeks after the deadline has expired.
%
%
%\paragraph{Remark \theremark}\stepcounter{remark} 
%will make a case for higher
%grade
% an analysis on the computational cost of the procedures
%implemented and details on the data structure used. 
%%The level of detail
%of the article~\cite{Andrade2008} might be taken as example for a high
%quality report.

%
%\item  The results of the experiments must be reported either in graphical form
%or in form of tables or both. Moreover, for the best solver resulting
%from the point 4, a table must be provided with the best results for
%each specific instance of Table~\ref{bench}.
%

%
%\item The report must contain enough details to guarantee the
%  reproducibility of the algorithm and experiments from the report alone (i.e., without
%   looking at the source code). 

\item The total length of the report should not be less than 5 pages and not
be more than 10 pages, appendix included (lengths apply to font size of
11pt and 3cm margins). Although these bounds are not strict, their
violation is highly discouraged.  In the description of the algorithms,
it is allowed (and encouraged) to use short algorithmic sketches in form
of pseudo-code but not to include source codes.


\item This is a list of factors that will be taken into account in the
evaluation:


\begin{itemize} 
\item quality of the final results;
\item level of detail of the study;
\item complexity and originality of the approaches chosen;
%\item originality of the experimental questions;
\item organization of experiments that guarantees reproducibility of
  conclusions;
\item clarity of the report;
\item presence of the analysis of the computational costs involved in
  the main operations of the local search.

\item effective use of graphics in the presentation of experimental results.
\end{itemize}


\item 
%In the project requirements 
%of Sec.~\ref{requirements} the words
%  ``one or more algorithms'' do NOT imply 
% ``the more algorithms, the better the final grade''. 
Note that a few, well thought algorithms are better than many naive ones!

\end{remarks}

\newpage

%\begin{appendices} 


\section{Instructions for the Submissions at the Web Page}
\label{a1}


Start early to test your submission. You can submit as many times as you
wish within the deadline. Every new valid submission overwrites the
previous submission.

Your archive must uncompress as follows:

\begin{itemize}
\item \texttt{Makefile} this file contains a ``clean'' goal that calls
  the clean of the Makefile inside the \texttt{src/} directory
\item \texttt{src/ } the source files that implement the algorithm
\item \texttt{src/Makefile} to compile the sources in \verb=src= and
 produce an executable called \verb=xyz=.
\item \texttt{src/make.local} this file must contain the paths to the
  libraries. It will be overwritten when you submit.
\item \texttt{bin/ } the executable to run the algorithm (copied from
  src/ by make)
\end{itemize}


Your program will be compiled calling \verb=make xyz=. Hence make sure
that your Makefile is present and that it has the task \verb=xyz=. 
Make sure that the file \texttt{src/Makefile} contains a line with   

\begin{verbatim}
 include ../../make/make.local
\end{verbatim}

in case you are using the boost, EasyLocal and Coin libraries.
The executable file \verb=xyz= generated in \verb=src= will be used for
the tests. Locally, you are recommended to have additionally the
following content, which however you do not have to submit:

\begin{itemize}
\item \texttt{data/ } containing the test instances.
\item \texttt{res/ } containing your results, the
  performance measurements
\item \texttt{r/ } statistics, data analysis, visualization
\item \texttt{doc/ } a description of the
  algorithms.
\item \texttt{log/ } other log files produced by the run of the
  algorithm
\end{itemize}


To prepare your submission create an archive as follows:
\begin{verbatim}
tar czvf bp.tgz Makefile src 
\end{verbatim}
Any other type of archive will not work.


The programs will run on a machine with ubuntu 14.04. 
%You can log in on
%any machine of the terminal room, compile your program there, and make
%sure that it executes.
%Other IMADA machines from the terminal room
%would also be good testing machines.

\medskip

The executable \verb=xyz= must run with the default arguments of
easylocal. In particular, the instance, the output file and
the seed will be specified.

\begin{verbatim}
xyz --main::instance ../data/toy.mps --main::output_file toy.sol --main::seed 1
\end{verbatim}

No other parameter can be specified. If you have any, then you will need
to hard code them. You can do this as follows. For main parameters just
add in the main file, for example:

\begin{verbatim}
method="BIFI";
\end{verbatim}

You will need to set the time limit by passing it to the Parametrized
instance of SimpleLocalSearch:

\begin{verbatim}
BP_solver.SetParameter("timeout",120.0);
\end{verbatim}

The program will be killed externally at the time limit if it did not
terminate already. Hence make sure you hard code the time limit and that
you output in time the best solution you found during the search.

All output must go in the standard output and in the output file. Do not
create other files.  The files must be in ASCII format and the output
file must contain the solution certificate in the format described in
the next section. Your solutions will be verified by an external program.

\section{Solution Format}
\label{a2}

   \textbf{Output:} There is no specific order in the solvers output
    lines. However, all lines, according to its first character, must belong
    to one of the three following categories:

    \begin{itemize}
    \item comments (any information that authors want to emphasize), beginning with the two chars ``c ``
    \item solution (satisfying all constraints or not). Only one line of this type is
      allowed. This line must begin with the two chars ``s `` and must be
      one of the following ones:
\begin{verbatim}
s FEASIBLE
s INFEASIBLE
\end{verbatim}
    \item values of a solution (if any), beginning with the two chars
      ``v `` (explained below).
    \end{itemize}
 

    The line starting with ``v `` must provide a 0-terminated sequence
    of  the indices of
the variables that are set to one. All variables whose indices are not
listed are set to zero. 
Arbitrary white
    space characters, including ordinary white spaces and tabulation
    characters, are allowed between the indices, as long as the line
    containing the indices is a value line, i.e. it begins with the
    two chars ``v ``.

    The certificate in a value line must be provided in both cases of a
    feasible and infeasible solution.

    Note that indices must start from 1 and not from 0. In the delivered
    starting package they start from 0.

For instance, the following output is valid for a given instance:

\begin{verbatim}
mycomputer:~$ ./mysolver myinstance-sat
c mysolver 6.55957 starting with SATTIMEOUT fixed to 120s
c Trying to guess a solution...
s FEASIBLE
c Objective 2000
v 3 4 6 18 21 1 7 0
c Time: 234s.
\end{verbatim}


The operator \verb!<<! has been overloaded for objects of type Output in
order to print a solution as described in this Appendix. 

%\end{appendices}

%\end{document}


\newpage
\begin{solution}


Let $K$ be the number of homogeneous vehicles and let's assume there is
only one single depot. Let $N=(V\cup S\cup E,A)$ be the network made by: 

\begin{itemize}
\item $V$  the set of vertices representing customers
\item $S$  the set of nodes representing the starting depots, one node
  for each of $k$ vehicless
\item $E$ the set of nodes representing the arrival depots, one node for
  each of $k$ vehicles
\item $A$ the set of arcs, we will assume the network to be symmetric. Hence for $uv\in A$, it is also $vu\in A$.
\end{itemize}
It is $|V\cup S\cup E|=n+2K$. The copies of the depots will allow us to
treat the routes as part of an unique giant tour.  We define the
following variables:
\begin{itemize}

\item $r_i$ the route to which each node $i \in V\cup S\cup E$ is assigned. The
  domain of the variable is $R=\{0..K_{UB}\}$

\item $l_k$ the total demand (load) in each route $k \in R$. The domain
  is $\{0..C\}$, where $C$ is the capacity of the vehicle.

\item $nr$ the number of routes $\{K_{LB}..K_{UB}\}$.

\item $s_i$ the succesor of a node $i$ in a route. The domain is
  $\{0..n+2K-1\}$.

\item $p_i$ the predecessor of a node $i$ in a route. The domain is
  $\{0..n+2K-1\}$.

\item $c_i$ the cumulated demand up to node $i$ in its route. The domain
  is $\{0..C\}$.
	
\item $t$ the total length of the giant tour. It has domain $\{0..M\}$,
  where $M$ is any large enough number.

\end{itemize}

Further we define the following parameters:
\begin{itemize}

\item the demand $q_i$ that is $0$ for $i\in S\cup E$ and equal to the demand
  corresponding to the costumer node $i$ for $i\in V$ 
\item the arc length $d_{ij}$ that is $0$ for $\{ij \mid i\in S\cup E, i
  \in S\cup E\}$ and equal to the length of the arc in the instance for
  the $\{ij \mid i\in V,j\in V\}$

\end{itemize}

The lower bound asked in point 1.~can be calculated as 
\[
K_{LB}=\left\lceil \frac{1}{C}\sum_{i\in S,E,V} q_i \right\rceil
\]
In order to find a valid value for $K$ as asked in point 2. and 3. we
can solve a binpacking problem. In gecode the global constraint
\verb=binpacking= takes care of this:
\begin{eqnarray}
\texttt{binpacking}(l, r, q)\\
(nr < j) \iff (l_{j - 1} = 0)) & \forall j = 1,\ldots,K_{UB}
\end{eqnarray}
The second constraint above takes care that all excess routes must be
empty. It can be reified as follows: 
\begin{eqnarray}
 \forall j = 1,\ldots,K_{UB} &  \texttt{BoolVar}\; b(0, 1)\\
&  \texttt{rel}( nr, \mbox{IRT\_LE}, j, b)\\
&  \texttt{rel}(l_{j - 1}, \mbox{IRT\_EQ}, 0, b)
\end{eqnarray}

\end{solution}
